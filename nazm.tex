% The first command declare this document to be of type 'article', the most common type in LaTeX.
\documentclass{article}

% The geometry package, if used, must be the first package. We specify the four margins of the documents since the default values are too wide.
\usepackage[top=1.25in,bottom=1.25in,left=1.00in,right=1.00in]{geometry}

% The fontspec package allows us to change the main font to Nafees Nastaleeq for Urdu type-setting.
\usepackage{fontspec}

% bidi stands for bi-directional and gives access to the \setRTL command which allows us to specify that the text will flow from Left to Right
\usepackage{bidi}

\pagestyle{empty}           % Specifies the use of a no-frills page style
\setmainfont[Script=Arabic,Scale=1.5]{Nafees Nastaleeq}         % Specify the main font. This must be installed in your system. Change the scale value to increase the default font size.
\parindent 0mm          % Declare that every paragraph will NOT start indented

\newcommand{\emptylines}[1]{\vspace{#1\baselineskip}}

\newlength{\widthOfMisra}
\setlength{\widthOfMisra}{0.4\textwidth}

\newcommand{\misra}[1]{\makebox[\widthOfMisra][s]{#1}\\}

\begin{document}        % The actual content of the document is about to start

\setRTL                 % The text will flow from Right to Left
\linespread{1.9}        % Increases the line-spacing

\begin{center}          % The text in the 'center' block will appear horizontally centered

   \begingroup
   \Large
      کتے
   \endgroup
   \emptylines{1}       % Adds a single empty line between the previous text and what follows

   (فیض احمد فیضؔ)
   \emptylines{3}       % Adds two empty lines worth of vertical space to separate the heading from the rest of the text


   \misra{یہ گلیوں کے آوارہ بیکار کتے}
   \misra{کہ بخشا گیا جن کو ذوقِ گدائی}
   \misra{زمانے کی پھٹکار سرمایہ اِن کا}
   \misra{جہاں بھر کی دھتکار اِن کی کمائی}

   \emptylines{3}
   \misra{نہ آرام شب کو نہ راحت سویرے}
   \misra{غلازت میں گھر، نالوں میں بسیرے}
   \misra{جو بگڑیں تو ایک دوسرے سے لڑا دو}
   \misra{ذرا ایک روٹی کا ٹکڑا دکھا دو}


\end{center}


\end{document}
